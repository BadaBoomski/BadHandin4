% !TeX root = ../../Handin4.tex
%*******************************************************************************
%	Figurer
%*******************************************************************************
\subsection{Figurer}
\label{sec:figurer}
Syntaks for at indsætte en figur (m. custom macro)
\begin{verbatim}
\insertfigure{billedenavn.png}{bredde}{figurlabel}
{Figurtekst}
\end{verbatim}

Parameter 1: Navn på billedefil. Skal ligge i kapitlets/sektionens \texttt{images/.} mappen

Parameter 2: Tal mellem 0 og 1 der specificerer bredden på billedet i forhold til sidebredden

Parameter 3: Label til figuren, benyttes til at referere til figuren fra tekst (skal være unikt, og helst give mening)

Parameter 4: Figurteksten 

\subsubsection{Figur med fuld bredde}
\begin{verbatim}
\insertfigure{eksempelfigur.png}{1}{fuldbredde}
{Figurtekst til figur med bredden 1}
\end{verbatim}

\insertfigure{eksempelfigur.png}{1}{fig:fuldbredde}
{Figurtekst til figur med bredden 1}

\subsubsection{Figur med 0.5 bredde}
\begin{verbatim}
\insertfigure{eksempelfigur.png}{0.5}{halvbredde}
{Figurtekst til figur med bredden 0.5}
\end{verbatim}

\insertfigure{eksempelfigur.png}{0.5}{halvbredde}
{Figurtekst til figur med bredden 0.5}

\subsubsection{Figur med 0.25 bredde}
\begin{verbatim}
\insertfigure{eksempelfigur.png}{0.25}{kvartbredde}
{Figurtekst til figur med bredden 0.25}
\end{verbatim}

\insertfigure{eksempelfigur.png}{0.25}{kvartbredde}
{Figurtekst til figur med bredden 0.25}

osv



\subsubsection{Referencer til Figurer}

\begin{verbatim}
Figuren fra forrige afsnit med halv bredde er figur \reffig{halvbredde}
\end{verbatim}
