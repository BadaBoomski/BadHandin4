% !TeX root = ../../Handin4.tex

\section{Design}

Der er til denne løsning benyttet 1 NoSQL-database og 2 SQL-databaser for at overholde kravet om mindst 1 af hver jf. kravsspec.

Gruppen har valgt at lade Traderdatabasen være NoSQL, da dokumentdatabasen blev vurderet som den mest fleksible, idet den ikke er så låst fast som SQL-databaserne.

\subsection{Entity Relationship Diagram (ERD)}

\subsection{Database Studio Design (DSD)}

\subsection{Domain Drive Design (DDD)}



\insertfigure{../../../images/diagrammer/DDD-prosumer-grid/ddd-diagram-prosumer-grid.png}{1}{ddd}{Opstilling af DDD-diagram af ProsumerInfo og SmartGridInfo.}
\insertfigure{../../../images/diagrammer/DDD-trader/ddd-diagram-trader.png}{1}{ddd}{Opstilling af DDD-diagram af TraderInfo.}


smartgridinfo:
-smartgrid
    info omkring dette grid, antal prosumers m.m, der vil kun være én række
-smartgridprsoumers
    info omkring prosumers for hvem der er i dette grip, installations nr m.m

prosumerinfo:
-prosumers:
    infoomkring total produktion af kWh blokke
    stamps for nyeste meter måling 1 række pr rosumer
-prosumersmartmeterrecordslog
    tidsstemplede elementer, med prosumer id, for hver gang der er sendt data fra en given elmåler/smartmeter


traderinfo:
-trader
    info omkring trader delen i forhold til grid, total brug af kWh fra plant(uden for systemet) eller hvis negativ viser den forbrug. 

-prosumertraderinfo
    en række pr prosumer, giver info om totalproduktion og profit og sellrate

-completed trades
    en ny række for hver enkel handel for en given periode
