% !TeX root = ../../Handin4.tex

\section{Design}

Der er til denne løsning benyttet 1 NoSQL-database og 2 SQL-databaser for at overholde kravet om mindst 1 af hver jf. kravsspec.

Gruppen har valgt at lade Traderdatabasen være NoSQL, da dokumentdatabasen blev vurderet som den mest fleksible, idet den ikke er så låst fast som SQL-databaserne.

\subsection{Entity Relationship Diagram (ERD)}

\subsection{Database Studio Design (DSD)}

\subsection{Domain Drive Design (DDD)}

I vores DDD har vi lavet 2 diagrammer, hvor der i det ene er vores Smartgrid og Prosumer, mens der i  den anden er vores Trader.

I det første diagram, SmartGrid er der 2 klasser, hhv. SmartGrid og SmartGridProsumer. SmartGrid indeholder udover en PK også information om den valgte smartgrid og hvor mange prosumers, der er.

SmartGridProsumers inderholder info omkring prosumers for hvem, der er i dette grid. Blandt andet deres InstallationsID og SmartMeterID.

I Prosumers findes ligeledes 2 klasser, hhv. Prosumers og ProsumersSmartMeterRecoresLog.
Den første indeholder udover et ProsumerID bl.a. info omkring total produktion og total forbrug af kWh-blokke. Den anden klasse indeholder tidsstemplede elementer med prosumerID for hver gang, der sendes data fra en given elmåler/smartmeter.
\insertfigure{../../../images/diagrammer/DDD-prosumer-grid/ddd-diagram-prosumer-grid.png}{1}{ddd}{Opstilling af DDD-diagram af ProsumerInfo og SmartGridInfo.}


I det andet diagram findes der noget flere klasser, da det er en del mere kompleks. Vi vil i det følgende kort beskrive hver klasse:
\begin{itemize}

\item Trader: info omkring trader-delen ift. grid, total brug af kWh fra plant og hvis den er negativ viser den forbrug.
\item ProsumerTraderInfo: Hver prosumers info omkring totalproduktion, profit og ”sellrate”.
\item CompletedTradesLog: En ny række for hver enkel handel for en given periode med køber, sælger og pris for, der er aftalt.
\item CurrentTrades: Igangværende trades, køberID, sælgerID samt antal kWh blokke og pris.
\item PlannedTrades: Fremtidige trades, køber- og sælgerID samt stamps fra og til og antal blokke.
\item ProsumerTradesSales: Liste over specielle tilbud fra en prosumer, det pågældende tidspunkt, antal blokke, bestemt af en tidsmarkør for hvornår folk kan købe til prisen. 
\item 
\end{itemize}




    
\insertfigure{../../../images/diagrammer/DDD-trader/ddd-diagram-trader.png}{1}{ddd}{Opstilling af DDD-diagram af TraderInfo.}

