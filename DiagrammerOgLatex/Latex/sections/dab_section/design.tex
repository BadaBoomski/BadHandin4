% !TeX root = ../../Handin4.tex

\section{Design}

Der er til denne løsning benyttet 1 NoSQL-database og 2 SQL-databaser for at overholde kravet om mindst 1 af hver jf. kravsspec.

Gruppen har valgt at lade Traderdatabasen være NoSQL, da dokumentdatabasen blev vurderet som den mest fleksible, idet den ikke er så låst fast som SQL-databaserne.

I forhold til den indledende fase og design har gruppen ikke valgt at bruge et ERD til at understøtte designet. Gruppen har i stedet i fællesskab tegnet systemet op i hånden og diskuteret, hvorledes klasserne og databaserne skulle opsættes. Det samme gjorde sig ligeledes gældende med Handin 3.2, del2, hvilket virkede fint.
Skulle man dog overdrage systemet til en anden udvikler eller skulle der ske hurtig refaktorering vil et ERD selvfølgelig være en fin hjælp at have. Og som med så mange andre diagrammer, styrker det forståelsen af det system som man arbejder med.

Gruppen har ligeledes hurtigt besluttet - med accept fra underviser Jesper Tørresø - at de 3 databaser ikke nødvendigvis skulle snakke sammen.
Dette er derfor ikke implementeret i systemet, men vil være en oplagt del at fokusere på i det fremtidige arbejde.

Den tekniske platform er udarbejdet på .NET Core med MVC- og Repository-pattern.
Der er benyttet Swagger til udarbejdelse af Rest-Api'er. Denne fremfor Postman, da gruppen har bedre erfaringer med Swagger.

Til gennemførelse af gruppens opgave er der taget stor inspiration fra medlemmernes løsning af Handin 3.2, del 1 og del 2.


