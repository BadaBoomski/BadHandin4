% !TeX root = ../../Handin4.tex

\section{Design}

Der er til denne løsning benyttet 1 NoSQL-database og 2 SQL-databaser for at overholde kravet om mindst 1 af hver jf. kravsspec.

Gruppen har valgt at lade Traderdatabasen være NoSQL, da dokumentdatabasen blev vurderet som den mest fleksible, idet den ikke er så låst fast som SQL-databaserne.

\subsection{Entity Relationship Diagram (ERD)}

\subsection{Database Studio Design (DSD)}

\subsection{Domain Drive Design (DDD)}

I vores DDD har vi lavet 2 diagrammer, hvor der i det ene er vores Smartgrid og Prosumer, mens der i  den anden er vores Trader.

I det første diagram, SmartGrid er der 2 klasser, hhv. SmartGrid og SmartGridProsumer. SmartGrid indeholder udover en PK også information om den valgte smartgrid og hvor mange prosumers, der er.

SmartGridProsumers inderholder info omkring prosumers for hvem, der er i dette grid. Blandt andet deres InstallationsID og SmartMeterID.

I Prosumers findes ligeledes 2 klasser, hhv. Prosumers og ProsumersSmartMeterRecoresLog.
Den første indeholder udover et ProsumerID bl.a. info omkring total produktion og total forbrug af kWh-blokke. Den anden klasse indeholder tidsstemplede elementer med prosumerID for hver gang, der sendes data fra en given elmåler/smartmeter.
\insertfigure{../../../images/diagrammer/DDD-prosumer-grid/ddd-diagram-prosumer-grid.png}{1}{ddd}{Opstilling af DDD-diagram af ProsumerInfo og SmartGridInfo.}


I det andet diagram findes der noget flere klasser, da det er en del mere kompleks. Vi vil i det følgende kort beskrive hver klasse:
\begin{itemize}

\item Trader: info omkring trader-delen ift. grid, total brug af kWh fra plant og hvis den er negativ viser den forbrug.
\item ProsumerTraderInfo: Hver prosumers info omkring totalproduktion, profit og ”sellrate”.
\item CompletedTradesLog: En ny række for hver enkel handel for en given periode med køber, sælger og pris for, der er aftalt.
\item CurrentTrades: Igangværende trades, køberID, sælgerID samt antal kWh blokke og pris.
\item PlannedTrades: Fremtidige trades, køber- og sælgerID samt stamps fra og til og antal blokke.
\item ProsumerTradesSales: Liste over specielle tilbud fra en prosumer, det pågældende tidspunkt, antal blokke, bestemt af en tidsmarkør for hvornår folk kan købe til prisen. 
\item 
\end{itemize}




    
\insertfigure{../../../images/diagrammer/DDD-trader/ddd-diagram-trader.png}{1}{ddd}{Opstilling af DDD-diagram af TraderInfo.}


smartgridinfo:
-smartgrid
    info omkring dette grid, antal prosumers m.m, der vil kun være én række
-smartgridprsoumers
    info omkring prosumers for hvem der er i dette grip, installations nr m.m

prosumerinfo:
-prosumers:
    infoomkring total produktion af kWh blokke
    stamps for nyeste meter måling 1 række pr rosumer
-prosumersmartmeterrecordslog
    tidsstemplede elementer, med prosumer id, for hver gang der er sendt data fra en given elmåler/smartmeter


traderinfo:
-trader
    info omkring trader delen i forhold til grid, total brug af kWh fra plant(uden for systemet) eller hvis negativ viser den forbrug. 

-prosumertraderinfo
    en række pr prosumer, giver info om totalproduktion og profit og sellrate

-completedtradeslog
    en ny række for hver enkel handel for en given periode, m. køber og sælger samt blokke og pris der er aftalt

-currenttrades
    igangværende trades, køber id og sælger id, samt pris og andtal kWh blokke, og fra og til stidsstamp

-plannedtrades
    fremtidige trades, køber id og sælger id, samt stamps fra og til, og antal blokke

-prosumertradessales
    liste over specielle tilbud fra en prosumer, om en bestemt tidspunkt og antal blokke( f.eks til en billigere pris om natten), bestemt af et tidsmarkør for hvornår folk kan købe til prisen

-prosumertradesoffers
    antal kWh blokke til salg og pris, men uden tidsperiode/interval